\section{Metodología de la investigación}
La metodología exploratoria aplicada es un enfoque de investigación que se utiliza para explorar y comprender fenómenos o problemas que son poco conocidos o no han sido investigados previamente de manera exhaustiva \citep{zafra2006}. Se emplea cuando existe una falta de información o teorías bien establecidas sobre el tema de estudio, como lo son la identificación y monitoreo de plagas en el cultivo de aguacate Hass a través de metodologías MLOps.

De acuerdo con Zafra (\citeyear{zafra2006}), la metodología exploratoria aplicada es flexible y adaptable, permitiendo a los investigadores ajustar su enfoque a medida que se obtiene nueva información. Al principio del proceso, el investigador puede tener una comprensión limitada del fenómeno, pero a medida que avanza la investigación, se van generando nuevas preguntas e ideas que guían el proceso de recolección y análisis de datos.

La metodología exploratoria aplicada fue un enfoque flexible y creativo que permitió a los investigadores en ingeniería de software explorar y comprender fenómenos como el Big Data y las metodologías MLOps que son relativamente nuevos para generar conocimiento y a establecer las bases para investigaciones futuras más rigurosas y proporciona una base sólida para futuras investigaciones y puede contribuir a generar explicaciones y enfoques más avanzados en relación con la producción agrícola.

El estudio se llevó a cabo en cuatro huertos comerciales de aguacate Hass que se encuentran ubicados en los municipios de Sotará y Timbío del departamento del Cauca, donde existe un impacto considerable de las plagas \textit{Stenoma catenifer} y \textit{heilipus lauri} \citep{zapata2022}. Esta situación de plagas analizada cuenta con una amplia base de datos fotográficas que serán procesadas para esta investigación.

De allí que se oriente la investigación con las siguientes fases metodológicas que permitan dar cumplimiento al proyecto:

\textbf{Fase I:} Para el objetivo de implementar técnicas de procesamiento de imágenes para extraer características relevantes y mejorar la capacidad del modelo de Machine Learning en el reconocimiento y detección de las plagas \textit{Stenoma catenifer} y \textit{heilipus lauri} en el cultivo de aguacate Hass a partir de imágenes capturadas en campo:
\begin{itemize}
    \item Recopilar datos e imágenes que permitan una base de datos para reconocer las características propias de plantas con y sin plagas.
    \item Procesar los datos e imágenes de la plantación de café donde se desarrolla el proceso de redimensionar las imágenes, normalizar los valores de los píxeles y elimina los posibles errores de imagen.
    \item Realizar técnicas de segmentación para aislar las regiones relevantes que contienen la planta de café.
\end{itemize}

\textbf{Fase II:} El objetivo es desarrollar y entrenar un modelo de Machine Learning utilizando técnicas apropiadas de preprocesamiento y selección de características, así como algoritmos de aprendizaje supervisado o no supervisado, para lograr una detección de las plagas \textit{Stenoma catenifer} y \textit{heilipus lauri} en el cultivo de aguacate Hass.
\begin{itemize}
    \item Dividir el conjunto de datos preprocesado en un conjunto de entrenamiento y un conjunto de prueba.
    \item Realizar la implementación de los pipelines necesarios para la automatización de las tareas de encapsulamiento y despliegue.
    \item Utilizar el conjunto de entrenamiento para entrenar el modelo seleccionado donde  el modelo asimilará el reconocimiento de los patrones asociados con y sin las plagas.
\end{itemize}

\textbf{Fase III:} En el objetivo desarrollar una metodología MLOps que permita la integración, automatización y monitoreo del modelo de Machine Learning diseñado para el reconocimiento y control de las plagas \textit{Stenoma catenifer} y \textit{heilipus lauri} en el cultivo de aguacate Hass:
\begin{itemize}
    \item Seleccionar el modelo de Machine Learning para la detección de objetos en imágenes.
    \item Utilizar técnicas como el filtrado, detección de bordes, extracción de texturas y características específicas de las plagas que desees detectar en el modelo.
\end{itemize}

\textbf{Fase IV:} En el objetivo validar el uso de MLOps mediante despliegue en un ambiente controlado, con la capacidad de monitorear y mejorar continuamente el rendimiento del modelo.
\begin{itemize}
    \item Evaluar el modelo utilizando el conjunto de prueba y ajustarlos a los parámetros y a la arquitectura del modelo según sea necesario para mejorar su rendimiento.
    \item Observar el rendimiento del conjunto de prueba para desplegarlo en tiempo real en aplicaciones de detección de plagas.
\end{itemize}

Todo este proceso se realizó aplicando los principios fundamentales de MLOps, siguiendo las medidas de seguridad informática requeridas para salvaguardar los datos personales, y además, cumpliendo con el conjunto de mejores prácticas en el desarrollo de software en general.