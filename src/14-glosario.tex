\section{Glosario de Términos}

\begin{description}
\item[Big Data:] Se refiere a los datos que son tan grandes o complejos que es difícil o imposible procesarlos con los métodos tradicionales.
\item[Machine Learning:] Sub-área de la inteligencia artificial que consiste en desarrollar algoritmos con la capacidad de aprender y mejorar a través de experiencias, sin necesidad de programarlas explícitamente.
%\emph{overfitting}.
\item[MLOps:] Es una extensión de la metodología DevOps que busca incluir los procesos de aprendizaje automático y ciencia de datos en la cadena de desarrollo y operaciones para hacer que el desarrollo del ML sea más confiable y productivo.
\item[Feature Store:] Validación y preparación de los datos, donde una capa de gestión de datos para el aprendizaje automático que permite compartir y descubrir funciones y crear canalizaciones de aprendizaje automático más eficaces.
\item[Código de fuente:] Colección de líneas de texto, escritas en un lenguaje de programación, que guían el proceso de ejecución de un programa. Estas instrucciones, que son comprensibles por humanos, están redactadas por un programador.
\item[Model Registry:] Modelo requerido para los atributos se les asigna un clave para distinguirlos de los demás registros. Relación (el vínculo que define la dependencia entre varias entidades).
\item[Pipeline:] Tuberías virtuales se crean para segmentar los datos y, de este modo, incrementar el rendimiento de un sistema digital.
\item[Plagas:] Cualquier ser vivo que resulta perjudicial para otro ser vivo de interés para el ser humano.

\end{description}