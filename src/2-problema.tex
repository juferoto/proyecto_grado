\section{Definición del problema}

\subsection{Planteamiento del problema}
La producción agrícola del aguacate Hass para el caso de México en el año 2019 correspondió a 2.4 millones de toneladas, aportando el 45\% en las exportaciones de este país y aumentando la cantidad de exportaciones en 22\% en el año 2020 \citep{cruz2022competitividad}. Estos aportes se reflejan en el PIB, contribuyendo al avance socioeconómico de las regiones agrícolas.

El aguacate Hass corresponde aproximadamente al 82\% de todos los aguacates el más consumido a nivel mundial y de acuerdo con los datos de la \citet{faostat2021hacia} el primer país productor de aguacate Hass es México con unas 2.393.849 toneladas al año, seguido de Colombia con unas 876.754 toneladas al año y de República Dominicana con 676.373 toneladas al año.

Los avances en la agroindustria en Colombia contribuyen al ámbito económico y laboral, siendo en la actualidad la producción agrícola del cultivo de aguacate Hass un producto de alta demanda a nivel nacional e internacional. Para el \citet{dane2016cultivo}, las problemáticas a tener en cuenta en el cultivo del aguacate Hass corresponden a: factores atmosféricos relacionados con la temperatura, las precipitaciones, el viento, la altitud, los factores de las condiciones del terreno y los factores relacionados con la siembra donde se encuentra la fertilización, los abonos y el tratamiento de las plagas y enfermedades.

Dentro de las plagas más preocupantes en el cultivo del aguacate Hass se encuentran la \textit{Stenoma catenifer} y el \textit{heilipus lauri}, insectos y larvas que introducen sus huevos provocando el daño en las semillas de los frutos en crecimiento. Además, el \textit{Stenoma catenifer} impacta en el fruto al perforar el brote terminal y los laterales del aguacate, formando túneles de hasta 25cm, corta los pedúnculos que son el pezón de la hoja y la base de los frutos pequeños, como resultado los frutos verdes y pequeños caen.

Para prevenir y controlar estas plagas, se recomienda implementar prácticas agrícolas adecuadas, como el manejo integrado de plagas, la selección de variedades resistentes y el control de la humedad en el suelo. Además, se deben realizar monitoreos constantes para detectar y tratar a tiempo cualquier plaga o enfermedad que pueda aparecer en el cultivo del aguacate Hass.

Para la detección de este tipo de plaga en la producción agrícola del cultivo del Hass existe el método Manejo Integrado de Plagas (MIP) en el cual se señala el monitoreo de manera manual y de observación constante partiendo de tres elementos, el primero corresponde a la prevención en relación a los cuidados, las restricciones y la limpieza del personal y sus utensilios de trabajo, el segundo al control donde se utiliza evaluaciones y registros manuales, instalación de trampas y el tercero es el manejo de la enfermedad en el cual se genera una protección y cuidado de las plantas dependiendo de los patógenos dañinos \citep{ica2012manejo}.

En este sentido, el Machine Learning permite a los científicos de datos, a través de imágenes, el reconocimiento de patrones de concentración y expansión de las plagas de manera óptima en todo el cultivo generando una reducción económica y mejorando la calidad del producto agrícola.

El cultivo del aguacate Hass en Colombia ha tenido una gran demanda a nivel nacional e internacional, generó un crecimiento del 34\% del total de área sembrada de aguacate. Además al ser un producto que presenta una cosecha constante por las condiciones del relieve y climáticas del país viene en un crecimiento de área sembrada de un 65\% anual desde el año 2019 \citep{proyectocolombiamide2021}. Dadas estas circunstancias, para los científicos de datos la implementación de la metodología MLOps se presenta como una oportunidad para mejorar la calidad, confiabilidad y eficiencia de los modelos de Machine Learning utilizados en la detección de plagas, evaluación del nivel de daño y reconocimiento de deformaciones y coloraciones específicas en las áreas afectadas. Al someter los modelos a rigurosos procesos de control de calidad, los científicos de datos con esta metodología garantizan la trazabilidad y transparencia a lo largo de todo el ciclo de vida del modelo, brindando así resultados más precisos y confiables.

\newpage
La utilización de MLOps se ha convertido en una práctica cada vez más extendida en el campo de la ciencia de datos, y se ha demostrado que mejora significativamente la eficiencia y la seguridad en la implementación de modelos de Machine Learning \citep{geron2019hands}. Su aplicación en el contexto de la agricultura y la predicción de plagas y enfermedades puede ser un paso importante para mejorar la productividad y sostenibilidad del cultivo de aguacate Hass y otros cultivos agrícolas.

Es importante destacar la relevancia de utilizar metodologías de MLOps para garantizar el correcto desarrollo, implementación y mantenimiento continuo de un modelo de control y cuidado de plagas permitiendo mejorar los procesos productivos agrícolas en Colombia. El modelo de Machine Learning al ser un programa orientado a la automatización y actualización constante de sus tareas avanza en el mejoramiento y la eficiencia de su procesamiento de información de manera continua a través de la metodología MLOps aportando a los procesos de análisis a los científicos de datos.

\subsection{Formulación del problema}

En este contexto la investigación busca desarrollar una herramienta digital que ayude a pronosticar y prevenir la presencia o no de las plagas como el \textit{Stenoma catenifer} y el \textit{heilipus lauri} en el cultivo de aguacate Hass, entendiendo que es crítica la detección temprana del brote en un cultivo, se propone la creación de un software accesible para los científicos de datos que les permita abordar esta problemática. Con base en esto, surge la siguiente pregunta de investigación: ¿Cómo el uso de la metodología MLOps en el desarrollo de un modelo de Machine Learning facilita la integración, la actualización y el despliegue continuo del reconocimiento de las plagas \textit{Stenoma catenifer} y \textit{heilipus lauri} en el cultivo de aguacate Hass, contribuyendo a mejorar los modelos agrícolas de forma automática y brindando beneficios económicos y sociales a la comunidad de científicos de datos? asimismo ¿Cómo mantener el programa de Machine Learning de forma automatizada y con supervisión continua, de modo que no se vea comprometido su rendimiento?