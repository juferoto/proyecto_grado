%%%%%%%%%%%%%%%%
% ABSTRACT
%%%%%%%%%%%%%%%%
\thispagestyle{empty}
\section*{Resumen}
Este estudio se enfocó en la implementación de una metodología MLOps en la agricultura, específicamente en el cultivo del aguacate Hass, que enfrenta desafíos como las plagas. La metodología MLOps se destaca por mantener la operación y el despliegue de modelos de aprendizaje automático mientras se mejora su rendimiento. El objetivo es desarrollar un modelo de Machine Learning para el reconocimiento y control de plagas, utilizando técnicas de preprocesamiento y selección de características. Se propuso la implementación de una metodología MLOps que permitió la integración, automatización y monitoreo del modelo ML, validándola en un entorno controlado. Se créo una herramienta digital para los científicos de datos, facilitando la predicción y prevención de plagas. El proyecto género un informe detallado del diseño, ejecución y evaluación de la metodología MLOps, así como la creación de una metodología que permita reevaluar continuamente el rendimiento del modelo de Machine Learning. Este enfoque contribuye a la sostenibilidad y productividad del sector agrícola.


\paragraph*{}{\textbf{Palabras Clave}} Big Data, Machine Learning, MLOps, Cultivo de
aguacate, Plagas Stenoma catenifer y heilipus lauri.

\section*{Abstract}
This study focused on the implementation of the MLOps methodology in agriculture, specifically in Hass avocado cultivation, which faces challenges such as pests. The MLOps methodology stands out for maintaining the operation and deployment of machine learning models while improving their performance. The objective is to develop a machine learning model for pest recognition and control, utilizing preprocessing techniques and feature selection. It was proposed to implement an MLOps methodology that allows for the integration, automation, and monitoring of the ML model, validating it in a controlled environment. The project aims to create a digital tool for data scientists, facilitating pest prediction and prevention. The project generated a detailed report on the design, execution, and evaluation of the MLOps methodology, as well as the creation of a methodology that enables continuous re-evaluation of the machine learning model's performance. This approach contributes to the sustainability and productivity of the agricultural sector.
\paragraph*{}{\textbf{Keywords}} Big Data, Machine Learning, MLOps, Hass avocado cultivation, Stenoma catenifer and heilipus lauri Pests.
