\thispagestyle{empty}
\begin{center}
    \Large{Ficha Resumen \\ Anteproyecto de Trabajo de Grado}
\end{center}

\textbf{Título: Metodología MLOps para la entrega continúa de un modelo de Machine Learning para el reconocimiento y control de las plagas Stenoma catenifer y heilipus lauri en el cultivo de aguacate Hass.}
\begin{enumerate}
    \item Área de trabajo: Ingeniería de Software
    \item Tipo de proyecto: Aplicado
    \item Estudiante: Juan Felipe Rodriguez
    \item Correo electrónico: jfrodriguezt@javerianacali.edu.co
    \item Dirección y teléfono: Carrera 112 48 - 92 Apto 701 Torre 5, 3058176473
    \item Director: Ph. D. David Arango
    \item Correo electrónico del director: david.arango@javerianacali.edu.co
    \item Palabras clave(al menos 5): Big Data, Machine Learning, MLOps, Cultivo de aguacate, Plagas Stenoma catenifer y heilipus lauri.
    \item Fecha de inicio: 23/Abril/2023
    \item Duración estimada (en meses): 6
    \item Resumen: La creciente aplicación de avances tecnológicos en diversos ámbitos de la vida ha llevado a la adopción de tecnologías innovadoras en la agricultura para mejorar la productividad y la eficiencia. Dentro de estas tecnologías, la metodología MLOps se destaca por su capacidad para mantener la operación de los modelos de aprendizaje automático y su despliegue, mientras se mejora y reentrenan los modelos, en donde se optimiza la toma de decisiones y el aumento de la precisión de los resultados.
\newpage
\thispagestyle{empty}

    La investigación se centra en el cultivo del aguacate Hass, un componente crucial para la economía y el desarrollo socioeconómico en países como México y Colombia. Sin embargo, este cultivo enfrenta desafíos significativos, como las plagas Stenoma catenifer y heilipus lauri. Para combatir este problema, se plantea una serie de objetivos que giran en torno a la implementación de la metodología MLOps y un modelo de Machine Learning. La metodología MLOps se presenta como una solución prometedora para combatir este problema, proporcionando un marco de trabajo apropiado para los científicos de datos, lo que les permite integrar, automatizar y monitorear los modelos de Machine Learning.

    En concreto, se busca implementar una metodología MLOps que permita la integración, automatización y monitoreo de un modelo de Machine Learning, específicamente diseñado para el reconocimiento y control de dichas plagas en el cultivo del aguacate Hass. Esta metodología seria validada mediante su despliegue en un entorno controlado, lo que permitiría monitorear y mejorar continuamente el rendimiento del modelo.

    Además, se planea desarrollar y entrenar este modelo de Machine Learning utilizando técnicas apropiadas de preprocesamiento y selección de características para garantizar una detección de las plagas en el cultivo de aguacate Hass. Con esto se tendría una herramienta digital accesible para los científicos de datos, que facilite la predicción y prevención de la aparición de plagas, lo que constituirá un recurso valioso para ellos.

    Finalmente, se espera que los resultados de este proyecto incluyan un informe detallado sobre el diseño, ejecución y evaluación de la metodología MLOps, así como la creación de una metodología MLOps que permita el monitoreo y reevaluación continua del rendimiento del modelo de Machine Learning a desarrollar. Este enfoque MLOps permitiría un seguimiento más adecuado del cultivo de aguacate Hass, contribuyendo así a la sostenibilidad y productividad del sector agrícola.
\end{enumerate}
