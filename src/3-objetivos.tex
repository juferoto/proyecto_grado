\section{Objetivos del proyecto}

\subsection{Objetivo General}
Implementar la metodología MLOps para la entrega continua de un modelo de Machine Learning para el reconocimiento y control de las plagas Stenoma catenifer y heilipus lauri en el cultivo de aguacate Hass.




\subsection{Objetivos específicos}
\begin{itemize}
  \item Implementar técnicas de procesamiento de imágenes para extraer características relevantes y mejorar la capacidad del modelo de Machine Learning en el reconocimiento y detección de las plagas Stenoma catenifer y heilipus lauri en el cultivo de aguacate Hass a partir de imágenes capturadas en campo.
  \item Desarrollar y entrenar un modelo de Machine Learning utilizando técnicas apropiadas de preprocesamiento y selección de características, así como algoritmos de aprendizaje supervisado o no supervisado, para lograr una detección de las plagas Stenoma catenifer y heilipus lauri en el cultivo de aguacate Hass.
  \item Desarrollar una metodología MLOps que permita la integración, automatización y monitoreo del modelo de Machine Learning diseñado para el reconocimiento y control de las plagas Stenoma catenifer y heilipus lauri en el cultivo de aguacate Hass.
  \item Validar el uso de MLOps mediante despliegue en un ambiente controlado, con la capacidad de monitorear y reevaluar continuamente el rendimiento del modelo.  
\end{itemize}

\newpage
\subsection{Resultados esperados}
En esta propuesta de investigación para la maestría en desarrollo de ingeniería de software se plantearon los siguientes resultados esperados:
\begin{enumerate}
  \item Implementación de un modelo de Machine Learning para detectar las plagas Stenoma catenifer y heilipus lauri en el cultivo de aguacate hass.
  \item La implementación de la estrategia MLOps que permitirá la integración, automatización y monitoreo continuo del modelo de Machine Learning. 
  \item Generación de una vista de despliegue que proporcione los componentes de la metodología MLOps.
  \item La demostración del modelo de Machine Learning utilizando la metodología MLOps, con la capacidad de monitorear y reevaluar continuamente el rendimiento del modelo.
\end{enumerate}