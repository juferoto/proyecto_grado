%%%%%%%%%%%%%%%%%
% Presentacion Anteproyecto
%%%%%%%%%%%%%%%%%
\thispagestyle{empty}
Santiago de Cali, \today

\newcommand{\underlinetext}[2][1pt]{{%
  \renewcommand{\ULdepth}{#1}%
  \uline{#2}%
}}

\begin{flushleft}
Ingeniero: \\
Juan Carlos Martínez Arias \\
Director Posgrados de Ingeniería \\
Facultad de Ingeniería y Ciencias \\
Pontificia Universidad Javeriana - Cali \\
\end{flushleft}


Con el fin de cumplir con los requisitos exigidos por la Universidad para llevar a cabo el Trabajo de Grado y posteriormente optar por el título de Magíster en Ingeniería, nos permitimos presentar a su consideración el proyecto de Trabajo de Grado denominado "\textit{Metodología MLOps para la entrega continúa de un modelo de Machine Learning para el reconocimiento y control de las plagas Stenoma catenifer y heilipus lauri en el cultivo de aguacate Hass}", el cual será realizado por el estudiante \textit{Juan Felipe Rodriguez Torres} con código \textit{3070801526} perteneciente al énfasis en Ingeniería de Software, bajo la dirección del profesor \textit{David Arango Londoño}.

El suscrito director del Trabajo de Grado autoriza para que se proceda a hacer la evaluación de este Proyecto ante el Tribunal que para el efecto se designe, toda vez que ha revisado cuidadosamente el documento y avala que ya se encuentra listo para ser presentado oficialmente.

\vspace{1cm}

\begin{table}[h]
\begin{tabular}{@{}p{0.5\textwidth} p{0.5\textwidth}@{}}
\textit{Firma Estudiante} \hrulefill & \textit{Firma Director} \hrulefill \\
\textit{Nombre Estudiante} \underlinetext[1pt]{Juan Felipe Rodriguez Torres} & \textit{Nombre Director} \underline{David Arango Londoño} \\
C.C. \underline{1130668332} \textit{de} \underline{Cali} & C.C. \underline{1130586950} \textit{de} \underline{Cali} \\
\end{tabular}
\end{table}

\vspace{1cm}
