\section{Alcance}
Los alcances que se desarrollaron en esta investigación correspondieron a una implementación de una metodología de MLOps para contribuir a un modelo de Machine Learning permitiendo la integración, la actualización y el despliegue continuo del reconocimiento de plagas \textit{Stenoma catenifer} y \textit{heilipus lauri} en el cultivo de aguacate Hass, partiendo de las siguientes fases:

\begin{itemize}
  \item \textbf{Experimentación / Desarrollo / Pruebas.}\\
    Feature Store: validación y preparación de los datos.\\
    Código de fuente.\\
    Model Registry.
  \item \textbf{Pre-Producción y Producción.}\\
    El modelo se despliega y se monitoriza.
    Pipelines
  \item \textbf{Fase de despliegue del modelo}
\end{itemize}

El desarrollo de una metodología MLOps que permita la implementación de modelos con mayor rapidez con procesos automatizados, contribuyendo a acelerar el tiempo de creación de valor al entregar información de manera ágil.

Fuera del alcance se encuentra la optimización del modelo de manera constante y la reutilización del modelo a medida que los datos evolucionan con el tiempo generando de manera efectiva mejoras al rendimiento y la adaptabilidad del modelo a partir de técnicas de transferencia de aprendizaje, donde se ajustan los pesos y las capas del modelo para adaptarse a los nuevos datos.

