\section{Introducción}

En los últimos años alrededor del mundo se viene implementando la metodología MLOps que permite la implementación y el despliegue eficiente y escalable de modelos de aprendizaje automático (Machine Learning) para los científicos de datos en diferentes entornos productivos como los agrícolas. Este tipo de intervención tecnológica se traduce en la capacidad de desarrollar modelos de predicción y análisis de datos agrícolas, como pronósticos climáticos, análisis de suelos, monitoreo de cultivos y detección temprana de enfermedades o plagas \citep{fao2021}.

El uso de los modelos de Machine Learning en la agricultura ofrece varias ventajas significativas, como por ejemplo permite aprovechar los datos recopilados de sensores y dispositivos IoT para tomar decisiones en datos y en tiempo real, asimismo optimizar el riego en el proceso de cosecha, la aplicación de fertilizantes y pesticidas y la planificación de la cosecha, que son aspectos clave para la producción agrícola necesarios para los científicos de datos en este ámbito económico.

Con la ayuda de la metodología MLOps, estos modelos propician la automatización de tareas repetitivas y complejas, como el procesamiento de grandes volúmenes de datos, la generación de informes y la gestión de la logística \citep{arleyllano2016,monsalve2021}, acciones que ahorran tiempo y recursos, permitiendo que los científicos de datos se enfoquen en actividades estratégicas y en la toma de decisiones de manera eficaz.

En este sentido, el MLOps para el aprendizaje automático, es un enfoque que combina prácticas y herramientas de desarrollo de software con técnicas de aprendizaje automático, brindando a los científicos de datos una serie de beneficios significativos como la automatización de tareas repetitivas, el entrenamiento y despliegue de modelos, además, facilita la colaboración entre los equipos de ciencia de datos y operaciones, promoviendo la comunicación fluida y el intercambio de conocimientos.\newpage

El MLOps proporciona a los científicos de datos una infraestructura sólida y procesos eficientes para desarrollar, implementar y mantener modelos de aprendizaje automático, en los distintos escenarios económicos y productivos, donde estas estrategias favorecen a las prácticas de control de versiones y monitoreo continuo, garantizando la trazabilidad y el control de calidad de los modelos.

La adopción de la metodología MLOps en la agricultura obedece a la capacidad para mejorar la calidad y la precisión de los resultados, ya que los modelos de aprendizaje automático pueden analizar patrones complejos en los datos y generar predicciones más precisas sobre el rendimiento de los cultivos, la salud de los suelos y otros aspectos agrícolas.
